\documentclass[a4paper, 12pt]{article} %%%here01

\usepackage[slovene]{babel}
\usepackage[utf8]{inputenc}
\usepackage[T1]{fontenc}
\usepackage{lmodern}
\usepackage{units}
\usepackage{eurosym}
\usepackage{amsmath}
\usepackage{amssymb}
\usepackage{amsthm}
\usepackage{amsfonts}
\usepackage{mathtools}
\usepackage{graphicx}
\usepackage{color}
%\usepackage{url}
\usepackage{hyperref}
\usepackage{enumerate}
\usepackage{enumitem}
\usepackage{pifont}

\usepackage[margin=0.5in]{geometry}

\DeclareMathOperator{\diam}{diam}
\DeclareMathOperator{\rank}{rank}

\newcommand{\D}{\mathbb{D}}
\renewcommand{\S}{\mathbb{S}}
\newcommand{\B}{\mathbb{B}}
\newcommand{\N}{\mathbb{N}}
\newcommand{\Z}{\mathbb{Z}}
\newcommand{\R}{\mathbb{R}}
\newcommand{\C}{\mathbb{C}}
\renewcommand{\P}{\mathbb{P}}

\newcommand{\closure}[1]{\overline{#1}}

\title{Analysis on manifolds - $2^{\text{nd}}$ homework}
\author{Benjamin Benčina, 27192018}

\begin{document}

\maketitle

\underline{\textbf{Ex. 1:}}
Let $F\colon\R^3\setminus\lbrace(0, 0, 0)\rbrace \to \R^6$ be such that $F(x_1, x_2, x_3) = (x_1^2, x_2^2, x_3^2, x_1 x_2, x_1 x_3, x_2 x_3)$.

\begin{enumerate}[label=(\alph*)]
	\item Let us first prove that $F$ is an immersion. We calculate the Jacobian matrix:
	\[ JF =
	\begin{bmatrix}
	2x_1 & 0 & 0 \\
	0 & 2x_2 & 0 \\
	0 & 0 & 2x_3 \\
	x_2 & x_1 & 0 \\
	x_3 & 0 & x_1 \\
	0 & x_3 & x_2
	\end{bmatrix}.
	\]
	For any element $(x_1, x_2, x_3)$ all three coordinates are not simultaneously equal to $0$. Without loss of generality, take $x_1 \neq 0$. The rows $[2x_1, 0, 0]$, $[x_2, x_1, 0]$, and $[x_3, 0, x_1]$ are clearly independent, hence $\rank JF = 3$ and $F$ is an immersion.
	
	\item Is $F$ an injective mapping? With the method of a sharp look we observe, that all coordinates in the image are quadratic in nature, meaning $F(x) = F(-x)$. As an example, take
	\begin{align*}
	F(1, 1, 1) &= (1, 1, 1, 1, 1, 1), \\
	F(-1, -1, -1) &= (1, 1, 1, 1, 1, 1).
	\end{align*}
	\item To determine the fibres of $F|_{\S^2}$, let $(x_1, x_2, x_3) \in \S^2$. Then $F^{-1}(F(x_1, x_2, x_3)) = \pm (x_1, x_2, x_3)$. These are precisely the antipodal points (also seen from the argument in (b)).
	
	Next, we would like to prove that $\widetilde{F}\colon\R\P^2 \to \R^6$, defined by
	\[
	\widetilde{F}(x_1:x_2:x_3) = \frac{F(x_1, x_2, x_3)}{x_1^2 + x_2^2 + x_3^2},
	\]
	is an injective immersion.
	We observe, that $\widetilde{F}(tx_1:tx_2:tx_3) = \widetilde{F}(x_1:x_2:x_3) = F(x_1, x_2, x_3)$, where $(x_1, x_2, x_3) \in \S^2$. This can be seen both by manipulating homogenous coordinates on the left hand side, or by simply eliminating $t$ from the expression on the right hand side of the defining equation. By (a), $\widetilde{F}$ is an immersion. Since only antipodal points on the sphere have the same image with $F$ and $\R\P^2$ identifies those points (as a qoutient of $\S^2$), the mapping is also injective.
\end{enumerate}

\underline{\textbf{Ex. 2:}}
Let $f\colon \R^3 \to \R$ be such that $f(x, y, z) = xy + z$ and let $M$ be the zero set of $f$. Let $V = -x \frac{\partial}{\partial x} - y \frac{\partial}{\partial y} + (z + 3xy) \frac{\partial}{\partial z}$ be a vector field.
\begin{enumerate}[label=(\alph*)]
	\item To prove that $M$ is a manifold, let us simply calculate the gradient $\nabla f = (y, x, 1)$, which can never be equal to $0$. Thusly, $M$ is a $2$-dimensional submanifold in $\R^3$.
	
	\item In order for $V$ to restrict to a vector field on $M$, $V$ has to be tangent to $M$. In $p=(x, y, z) \in M$, we calculate
	\[
	df_p(V) = V(f(x, y, z)) = \frac{\partial f}{\partial x}(-x) + \frac{\partial f}{\partial y}(-y) + \frac{\partial f}{\partial z}(z + 3xy) = y(-x) + x(-y) + 1(2xy) = -2xy + 2xy = 0.
	\]
	
	\item Denote by $W$ the restriction of $V$ to $M$. We are searching for its fixed points. In order to find them, let us calculate the flow of $W$, that is, we will solve
	\[
	\frac{\partial}{\partial t}\varphi_W(t, x, y, z) = W(\varphi_W(t, x, y, z)).
	\]
	Equivalently, we have
	\[
	\dot{x}(t)\frac{\partial}{\partial x} + \dot{y}(t)\frac{\partial}{\partial y} + \dot{z}(t)\frac{\partial}{\partial z} = -x \frac{\partial}{\partial x} + -y \frac{\partial}{\partial y} + 2xy \frac{\partial}{\partial z}.
	\]
	We solve this in coordinates and get
	\begin{align*}
	x(t) &= Ae^{-t} \\
	y(t) &= Be^{-t} \\
	z(t) &= -ABe^{-2t}
	\end{align*}
	Using the initial condition $\varphi_W(0, x, y, z) = (x, y, z)$, we get
	\begin{align*}
	x(t) &= x_0e^{-t} \\
	y(t) &= y_0e^{-t} \\
	z(t) &= -x_0y_0e^{-2t}
	\end{align*}
	There is clearly only one fixed point, that is $(x, y, z) = (0, 0, 0)$.
	
	We now ask ourselves whether this fixed point is locally stable.
	Take a basis of neighbourhoods around $(0, 0, 0)$ as $U_n = \frac{1}{n}\B^3 \cap M$ for $n \in \N$. For $t \geq 0$ and $(x, y, z) \in U_n$, we calculate
	\[
	\max_{t\geq 0} |\varphi_W(t, x, y, z)| = |(x, y, -xy)| \implies \varphi_W(U_n) \subseteq U_n.
	\]
	The fixed point is therefore locally stable.
\end{enumerate}

\underline{\textbf{Ex. 3:}}
Let $M = \lbrace((x, y), [v: w]) \in \C^2 \times \C\P^1 ; \; xv = yv \rbrace$ and let $\pi\colon M \to \C^2$ be such that $\pi((x, y),[v:w]) = (x, y)$.
\begin{enumerate}[label=(\alph*)]
	\item Firstly, let us prove that $M$ is a complex manifold. Take $U_w = \lbrace((x, y), [v: w]) \in M ; \; w \neq 0 \rbrace$. On $U_w \subset M$ we now have
	\[
	((x, y),[\frac{v}{w}:1]) = ((\frac{v}{w}y, y), [\frac{v}{w}: 1]) \mapsto (y, \frac{v}{w} = z) \in \C^2.
	\]
	Similarly, on $U_v \subset M$ we have
	\[
	((x, y),[1:\frac{w}{v}]) = ((x, \frac{w}{v}x), [1:\frac{w}{v}]) \mapsto (x, \frac{w}{v} = z) \in \C^2.
	\]
	All that remains is to calculate the transition maps. Suppose $z \neq 0$ and calculate
	\begin{align*}
	\varphi_{wv}\colon & (x, z) \mapsto ((x, zx), [1:z]) = ((yz^{-1}, y), [z^{-1}:1]) \mapsto (y, z^{-1}) = (xz, z^{-1}), \\
	\varphi_{vw}\colon & (y, z) \mapsto ((zy, y),[z:1]) = ((x, xz^{-1}),[1:z^{-1}] \mapsto (x, z^{-1}) = (yz, z^{-1}),
	\end{align*}
	which are both holomrphic maps, since they are holomorphic on each component.
	
	\item Secondly, let us prove that the map $\pi|_{M \setminus \pi^{-1}(0, 0)}$ is a biholomorphism onto $\C^2\setminus\lbrace(0, 0)\rbrace$. Without loss of generality, suppose $x \neq 0$. On $M \setminus \pi^{-1}(0, 0)$ we now have
	\[
	\frac{w}{v} = \frac{y}{x}.
	\]
	The inverse function is now
	\[
	(\pi|_{M \setminus \pi^{-1}(0, 0)})^{-1}(x, y) = ((x, y), [\frac{y}{x}:1]),
	\]
	which is well-defined because of homogeneity of the second coordinate pair. This map is therefore bijective. Since it is a canonical projection, it is clearly holomorphic, and by extension biholomorphic.
	\item Lastly, we will show that the map $p\colon M \to \C\P^1$ given by $\pi((x, y), [v:w]) = [v:w]$ is a holomorphic line bundle over $\C\P^1$.
	\begin{itemize}
		\item Suppose $v \neq 0$. We have
		\[
		p^{-1}([v:w]) = ((x, y),[v:w]) = ((x, x\frac{w}{v}), [v:w]).
		\]
		This is a $1$-parameter family. In other words, the map $\varphi\colon\C\to p^{-1}([v:w])$ given by $x \mapsto ((x, x \frac{w}{v}), [v:w])$ is a vector space isomorphism (since $v, w$ are parameters, every component is linear), making $p^{-1}([v:w])$ a vector space of $\C$-dimension $1$. The same thought process holds for the case $w \neq 0$.
		\item Take $U_v = \lbrace [v:w] ; \; v \neq 0 \rbrace$ and $U_w = \lbrace [v:w] ; \; w \neq 0 \rbrace$. Consider the map $\gamma_v \colon p^{-1}(U_v) \to U_v \times \C$ given by $\gamma_v((x, y), [v:w]) = ([1:\frac{w}{v}], x)$. Let us calculate the preimage
		\[
		\gamma_{v}^{-1}([1:\frac{w}{v}], x) = ((x, \frac{w}{v}), [1:\frac{w}{v}]).
		\]
		We get
		\[
		\gamma_v\circ\gamma_v^{-1} = id,
		\]
		and same holds for $U_w$. We got a local trivialization.
		\item Again, what remains is to find the transition maps and verify that they are $\C$-linear. We calculate
		\[
		\gamma_{vw} = \gamma_{v}\circ\gamma_{w}^{-1}([\frac{v}{w}:1], y) = \gamma_v((\frac{w}{v}y, y), [v:w]) = ([1:\frac{w}{v}], \frac{w}{v}y),
		\]
		and similarly for $\gamma_{wv}$.
		The transition maps are in fact $\C$-linear and the transition coefficients are
		\[
		g_{vw} = \frac{v}{w},
		\]
		meaning this line bundle is equivalent to the tautological bundle $\mathcal{O}(-1)$, that is, its power is $k = 1$.
	\end{itemize}
\end{enumerate}

\underline{\textbf{Ex. 4:}}
Consider the differential $1$-form $\omega = A(x, y, z)dx + B(x, y, z)dy + C(x, y, z)dz$ on $\R^3$, where $A, B, C \colon \R^3 \to \R$ are smooth functions without common zeros. To simplify, we will write functions without their arguments, bar the first time they are written.

\begin{enumerate}[label=(\alph*)]
	\item Let us first show that $\xi = \ker \omega$ is a $2$-dimensional vector bundle over $\R^3$. Take an arbitrary derivation $V = \alpha(x, y, z)\frac{\partial}{\partial x} + \beta(x, y, z)\frac{\partial}{\partial y} + \gamma(x, y, z)\frac{\partial}{\partial z}$ and fix a point $p \in \R^3$. In $p$, we calculate
	\[
	\omega_p(V) = A\alpha + B\beta + C\gamma |_p = A(p)\alpha(p) + B(p)\beta(p) + C(p)\gamma(p).
	\]
	Since $A$, $B$, and $C$ do not have a common zero, at least one of them does not have a zero in $p$. Without loss of generality, let $C(p) \neq 0$. We can now express the coefficient $\gamma$ in the following way
	\[
	\gamma(p) = -\frac{A(p)\alpha(p) + B(p)\beta(p)}{C(p)}.
	\]
	For every $p \in \R^3$, $\ker\omega_p$ has $2$ dimensions and is a vector subspace of $T_p\R^3$, making $\ker\omega = \bigsqcup_{p\in\R^3}\ker\omega_p$ a vector subbundle over $\R^3$.
	
	\item Suppose $C \equiv 1$, and $A$ and $B$ are do not depend on $z$. We will find the conditions on $A$ and $B$ such that $\xi$ is totally integrable.
	
	Recall that by Frobenius' theorem, $\xi$ is totally integrable if and only if $\xi$ is an involutive subbundle, that is, for every point $p \in \R^3$ there has to exist $v_1, v_2$ vector fields in a neighbourhood of $p$, so that $v_1$ and $v_2$ span $\xi$ and $[v_1, v_2]$ is tangent to $\xi$. Note, that this theorem is independant of the choice of $v_1, v_2$, so we can take any two vector fields, that satisfy the conditions.
	
	With that in mind, take $v_1 = \frac{\partial}{\partial x} - A\frac{\partial}{\partial z}$ and $v_2 = \frac{\partial}{\partial y} - B\frac{\partial}{\partial z}$ as they clearly span $\xi$. Let us calculate the Lie bracket, cancelling second order parts as we go. Also note that $A_z = B_z = 0$ by assumption.
	\begin{align*}
	[v_1, v_2] &= [\frac{\partial}{\partial x} - A\frac{\partial}{\partial z}, \frac{\partial}{\partial y} - B\frac{\partial}{\partial z}] = [\frac{\partial}{\partial x}, \frac{\partial}{\partial y}] + [-A \frac{\partial}{\partial z}, \frac{\partial}{\partial y}] + [\frac{\partial}{\partial x}, -B \frac{\partial}{\partial z}] + [A\frac{\partial}{\partial z}, B\frac{\partial}{\partial z}] \\
	& = (A_y - B_x)\frac{\partial}{\partial z}.
	\end{align*}
	Now, insert it into $\omega$:
	\[
	\omega((A_y - B_x)\frac{\partial}{\partial z}) = A_y - B_x = 0.
	\]
	The condition we are searching for is
	\[
	A_y = B_x.
	\]
	
	\item Additionally assume that there exists a function $F \colon \R^3 \to \R$ such that $dF = \omega$. Let us find it's integral manifolds of $\xi$.
	
	The only thing to notice is that on $\xi = \ker\omega$ the equation transforms into $dF = 0$. Integrating both sides we get $F = c$ for constant $c \in \R$. In other words, $F$ is the integral of $\xi$ (in the sense of the theory of differential equations) and the integral manifolds must have the form $F^{-1}(c)$ for every $c \in \R$.
	
	\item Concretely, let $\omega = dz + xdy$. We will find the basis of $\xi$ and show that their commutator (Lie bracket) does not belong to $\xi$.
	
	Clearly, for the basis we can take $v_1 = \frac{\partial}{\partial x}$ and $v_2 = x\frac{\partial}{\partial z} - \frac{\partial}{\partial y}$. Let us calculate
	\[
	[v_1, v_2] = [\frac{\partial}{\partial x}, x\frac{\partial}{\partial z} - \frac{\partial}{\partial y}] = [\frac{\partial}{\partial x}, x\frac{\partial}{\partial z}] - [\frac{\partial}{\partial x}, \frac{\partial}{\partial y}] = \frac{\partial}{\partial x}(x\frac{\partial}{\partial z}) = \frac{\partial}{\partial z}.
	\]
	Now, insert it into $\omega$
	\[
	\omega(\frac{\partial}{\partial z}) = 1 \implies [v_1, v_2] \notin \xi.
	\]
\end{enumerate}

\end{document}