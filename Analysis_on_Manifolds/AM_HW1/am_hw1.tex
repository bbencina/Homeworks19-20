\documentclass[a4paper, 12pt]{article} %%%here01

\usepackage[slovene]{babel}
\usepackage[utf8]{inputenc}
\usepackage[T1]{fontenc}
\usepackage{lmodern}
\usepackage{units}
\usepackage{eurosym}
\usepackage{amsmath}
\usepackage{amssymb}
\usepackage{amsthm}
\usepackage{amsfonts}
\usepackage{mathtools}
\usepackage{graphicx}
\usepackage{color}
%\usepackage{url}
\usepackage{hyperref}
\usepackage{enumerate}
\usepackage{enumitem}
\usepackage{pifont}
\usepackage{tikz-cd}
\usetikzlibrary{babel}
\usepackage{adjustbox}

\usepackage[margin=0.5in]{geometry}

\newcommand{\N}{\mathbb{N}}
\newcommand{\R}{\mathbb{R}}
\newcommand{\C}{\mathbb{C}}

\newcommand{\closure}[1]{\overline{#1}}

\title{Analysis on Manifolds - $1^{\text{st}}$ homework}
\author{Benjamin Benčina, 27192018}

\begin{document}

\maketitle

\underline{\textbf{Ex. 1:}}
We want to find explicit smooth charts for the special linear group $SL_2(\R)$ of real matrices with determinant equal to $1$. For the remainder of this exercise we will make the identification
\[
\begin{bmatrix}
a & b \\
c & d
\end{bmatrix}
\longleftrightarrow (a, b, c, d)
\]
for ease of writing. Now we have $SL_2(\R) = \lbrace (a, b, c, d) ; \; ad-bc=1\rbrace$.

Firstly, $SL_2(\R)$ is obviously a Hausdorff and $2^{nd}$-countable space, since $SL_2(\R) \subset \R^4$ is a topological subspace with the standard topology, and hence inherits these properties.

Secondly, it is clear that if $SL_2(\R)$ is a manifold, its dimension is $3$, since one coordinate can be expressed with the other three. So, the principal idea is that if $a \neq 0$, we have $ad-bc = 1 \iff d = \frac{1+bc}{a}$. Take the set $U_a = \lbrace (a, b, c, d) \in SL_2(\R) ; \; a \neq 0 \rbrace = \lbrace (a, b, c, d) \in \R^4 ; \; a \neq 0 \rbrace \cap SL_2(\R)$, which is an open set in $SL_2(\R)$, since the set $\lbrace (a, b, c, d) \in \R^4 ; \; a \neq 0\rbrace$ is open in $\R^4$. On $U_a$ we now have points of form $(a, b, c, \frac{1+bc}{a})$. By denoting $d = d(a, b, c) = \frac{1+bc}{a}$, it is clear that the points $(a, b, c, d(a, b, c))$ form a graph of function $d$ above $\R_{a, b, c}^3\setminus\lbrace a = 0\rbrace$. Thus, the mapping $\varphi_a(a, b, c, d) = (a, b, c)$ constitutes a chart $(U_a, \varphi_a)$.
Note that for connected charts we simply split the two cases $a > 0$ and $a < 0$.

Next, we take the open set $U_b = \lbrace (a, b, c, d) \in SL_2(\R);\; b \neq 0\rbrace$. By denoting $c = c(a, b, d) = \frac{ad-1}{b}$ on $U_b$ it is again clear that the points $(a, b, c(a, b, d), d)$ form a graph of function $c$ above $\R_{a, b, d}^3\setminus\lbrace b = 0 \rbrace$. Thus, the mapping $\varphi_b(a, b, c, d) = (a, b, d)$ constitutes a chart $(U_b, \varphi_b)$.

We see that $U_a$ and $U_b$ cover $SL_2(\R)$, since the remaining case is one where $a = b = 0$, making the determinant equal to $0$. We only need to verify that both transition functions are smooth:
\begin{align*}
(\varphi_a\circ\varphi_b^{-1})(a, b, d) &= \varphi_a(a, b, \frac{ad-1}{b}, d) = (a, b, \frac{ad-1}{b}),\\
(\varphi_b\circ\varphi_a^{-1})(a, b, c) &= \varphi_b(a, b, c, \frac{1+bc}{a}) = (a, b, \frac{1+bc}{a}).
\end{align*}
Both are clearly smooth maps and well-defined on $\R^3\setminus\lbrace a = b = 0\rbrace$. We have verified that $SL_2(\R)$ is indeed a smooth $3$-dimensional real manifold.
\newline

\underline{\textbf{Ex. 2:}}
For each $n \in \N$, consider the map $\varphi_n\colon \R \to \R$ given by
\[
\varphi_n(t) = \begin{cases}
t; \; t \leq 0 \\
t^n; \; t > 0
\end{cases}
\]
\begin{enumerate}[label=(\alph*)]
\item
We want to prove that for every $n \in \N$ the map $\varphi_n$ is a homeomorphism. Bijectivity of $\varphi_n$ is clear, since $t \mapsto t$ is bijective on $\R_{-}\cup\lbrace 0 \rbrace$, $t \mapsto t^n$ is bijective on $\R_{+}$ and they concatenate at $0$. We repeat the same argument for continuity and continuity of the inverse
\[
\varphi_n^{-1}(t) = \begin{cases}
t; \; t \leq 0 \\
\sqrt[n]{t}; \; t > 0
\end{cases}.
\]
It is worth noting that this would be even more topologically obvious if we define the mappings equivalently as follows:
\[
\varphi_n(t) = \begin{cases}
t;\; t \leq 0 \\
t^n; \; t \geq 0
\end{cases}
\]
\item
We now want to determine which values $m,n \in \N$ are such that $\lbrace(\varphi_n, \R), (\varphi_m, \R)\rbrace$ is a smooth atlas for $\R$. We suppose that $n \neq m$, otherwise we really only have one chart. Let's calculate the transition maps:
\[
\varphi_{n,m}(t) = (\varphi_n\circ\varphi_m^{-1})(t) = \begin{cases}
t; \; t \leq 0 \\
t^{\frac{n}{m}}; \; t > 0
\end{cases}
\]
Since they are defined on the whole of $\R$, the problem is of course smoothness at $t = 0$. We calculate the left and right derivatives at $0$:
\begin{align*}
t \leq 0:& \lim_{t \to 0^{-}} \varphi_{n,m}'(t) = \lim_{t \to 0^{-}} 1 = 1 \\
t > 0:& \lim_{t \to 0^{+}} \varphi_{n,m}'(t) = \lim_{t \to 0^{+}} \frac{n}{m}t^{\frac{n}{m}-1} = \begin{cases}
0; \; \frac{n}{m} > 1 \\
\infty; \; \frac{n}{m} < 1
\end{cases}
\neq 1
\end{align*}
Hence, $n=m$ is the only case where the above atlas is a smooth atlas for $\R$.
\item
Let $\R_n = \lbrace (\varphi_n, \R) \rbrace$ be a smooth manifold. For which $n, m \in \N$ are $\R_n$ and $\R_m$ diffeomorphic? Again we suppose $n \neq m$, since $n=m$ is the trivial case.
Consider the following diagram:
\adjustbox{scale=1, center}{
\begin{tikzcd}
\R_n \arrow[d, "\varphi_n"] \arrow[r, "\Phi"] & \R_m \arrow[d, "\varphi_m"] \\
\R \arrow[r, "\widetilde{\Phi}"] & \R
\end{tikzcd}
}
By definition, the map $\Phi \colon \R_n \to \R_m$ will be a differentiable precisely when the map $\widetilde{\Phi} = \varphi_m \circ \Phi \circ \varphi_n^{-1} \colon \R \to \R$ is a differentiable map. With this in mind, we can simply choose a diffeomorphism $\widetilde{\Phi}$ and the corresponding map $\Phi = \varphi_m^{-1} \circ \widetilde{\Phi} \circ \varphi_n$ will be a diffeomorphism between manifolds. We will of course check the necessary properties.

Choose $\widetilde{\Phi} = id_\R$, an obvious diffeomorphism of the real line. Thus, we have
\[
\Phi = \varphi_m^{-1} \circ id_\R \circ \varphi_n = \varphi_m^{-1} \circ \varphi_n = \begin{cases}
t^{\frac{n}{m}} ; \; t > 0 \\
t ; \; t \leq 0
\end{cases}
\]
Indeed, by the argument from (a), it is bijective, and differentiable in the sense of manifolds, since $\widetilde{\Phi}$ is differentiable. Same holds for the inverse:
\[
\Phi^{-1} = \begin{cases}
t^{\frac{m}{n}} ; \; t > 0 \\
t ; \; t \leq 0
\end{cases}
\]
We have found a diffeomorphism between $\R_n$ and $\R_m$ for all $n, m \in \N$.
\end{enumerate}

\underline{\textbf{Ex. 3:}}
Let $n \in \N$ be a natural number and let $SO(n)$ be a special orthogonal group of $n \times n$ matrices. Consider the map given by
\[
\varphi\colon SO(n+1) \times \R^{n+1} \to \R^{n+1}, \; (A, x) \mapsto Ax.
\]
\begin{enumerate}[label=(\alph*)]
\item
To prove that $\varphi$ is a group action is trivial, since it is merely matrix multiplication.
Indeed, for every vector $x \in \R^{n+1}$ we have $I x = x$, matrix $I$ being the identity matrix, and because matrix multiplication is associative, for every $A, B \in SO(n+1)$ and $x \in \R^{n+1}$ we have $A(Bx) = (AB)x$.
\item
Next, we wish to see that $\varphi$ restricts to a well-defined action on the sphere $S^n \subset \R^{n+1}$. We have already seen that it is indeed an action, all that remains is to show that $\varphi(SO(n+1), S^n) \subseteq S^n$. But this is easy taking into account a well-known fact from algebra that special orthogonal matrices preserve the scalar product, and hence the norm (or length) of vector. That is, for every $A \in SO(n+1)$ and $x, y \in \R^{n+1}$ we have that $Ax \cdot Ay = x \cdot y$. Now, by definition, $||Ax||^2 = Ax \cdot Ax = x \cdot x = ||x||^2$. Since elements of $S^n$ are precisely vectors with unitary norm, the conclusion follows.
\item
To determine the isotropy group $G$ of $\varphi$ at the vector $(1, 0, \dots, 0) \in S^n$, all we need to do is remember that matrix columns are images of base vectors:
\[
G = \lbrace A \in SO(n+1) ; \; A(1, 0, \dots 0) = (1, 0, \dots, 0) \rbrace = \lbrace A \in SO(n+1) ; \; A^{(1)} = (1, 0, \dots, 0) \rbrace.
\]
\item
Finally, we want to prove that the group quotient $SO(n+1)/G$ is a smooth manifold diffeomorphic to $S^n$. Denote by $\pi \colon SO(n+1) \to SO(n+1)/G$ a quotient map, defined by $A \mapsto AG$.

Firstly, $2^{\text{nd}}$-countability is obvious, since it is a quotient property.

Secondly, to prove that $SO(n+1)/G$ is Hausdorff, we first need to see that the quotient map $\pi$ is open. Take an open set $V \subset SO(n+1)$. The image $\pi(V)$ will be open in the quotient precisely when the set $\pi^{-1}(\pi(V))$ is open in $SO(n+1)$. Let's calculate:
\[
\pi^{-1}(\pi(V)) = \pi^{-1}(VG) = \pi^{-1}(\lbrace AG ; \; A \in V \rbrace) = \lbrace AB ; \; A \in V, \; B \in G \rbrace = VG.
\]
Since $SO(n+1)$ is a topological group and $V$ is an open set, $VG$ is open in $SO(n+1)$.
Now, Hausdorffness is equivalent to the quotient relation being closed, that is, we want the set $\Delta = \lbrace (A, B) ; \; AG = BG \rbrace$ to be closed in $SO(n+1) \times SO(n+1)$. Let's calculate again:
\[
\Delta = \lbrace (A, B) ; \; AG = BG \rbrace = \lbrace (A, B) ; \; B^{-1}A \in G \rbrace = F^{-1}(G),
\]
where $F(A, B) = B^{-1}A$ is a continuous function. Since $G$ is closed in $SO(n+1)$ (we merely fixed some components) and $F$ is continuous, $\Delta$ is closed in $SO(n+1) \times SO(n+1)$.

Thirdly, let's further inspect cosets in $SO(n+1)/G$.
\begin{align*}
AG = BG &\iff B^{-1}A \in G \iff B^{-1}A (1, 0, \dots 0) = (1, 0, \dots, 0)\\
 &\iff A(1, 0, \dots, 0) = B(1, 0, \dots, 0) \iff A^{(1)} = B^{(1)}
\end{align*}
Cosets are thus uniquely determined by where they send the vector $(1, 0, \dots, 0)$, that is, by their first column. But our matrices are in $SO(n+1)$, meaning their columns are orthonormal, and hence of unitary norm. In other words, for every $A \in SO(n+1)$ the vector $A^{(1)}$ lies in $S^n$.
We now define an identification map
\[
\psi\colon SO(n+1)/G \to S^n, \; AG \mapsto A^{(1)} =: v_A.
\] By previous calculation, this is a well-defined bijection (that it is truly surjective is clear, since every vector of unitary norm gives a family of special orthogonal matrices composed by this vector in the first column and the orthonormal basis of its orthogonal complement space in the other columns).

We will now show that $\psi$ is in fact a homeomorphism. Since it is a bijection and clearly continuous (one way to look at it is as a projection to the first column), it is enough to show that it is an open map. We verify this on basic open sets in $SO(n+1)$, which we get by taking open balls in $SO(n+1)$ and mapping them by $\pi$. That is, take a matrix $A \in SO(n+1)$ and an open ball $\mathbb{B}(A, \epsilon)$ of radius $\epsilon$ around $A$. In particular, projecting $\mathbb{B}(A, \epsilon)$ to the first column (really just intersecting with $\R^{n+1}$ at the first column) we clearly see that this is also a ball around $A^{(1)}$ of radius $\epsilon$ (as an example of this thought exercise imagine intersecting $\mathbb{B}^2$ with the real line and getting $\mathbb{B}^1$). Now, since $\pi$ is an open and continuous mapping, it is clear that $\pi(\mathbb{B}(A, \epsilon))$ is an open ball of radius $\epsilon$ around the coset $AG$. Mapping this ball with our identification map $\psi$ gives us an open ball or radius $\epsilon$ around $v_A = A^{(1)} \in S^n$. The map $\psi$ is therefore a homeomorphism, and hence $SO(n+1)/G$ is a topological manifold with charts from $S^n$ mapped back by $\psi^{-1}$, that is, if $\lbrace (U_i, \varphi_i) \rbrace$ is an atlas on $S^n$, we define $\lbrace (\psi^{-1}(U_i), \phi_i = \varphi_i\circ\psi)\rbrace$ to be an atlas on $SO(n+1)/G$.

The last step is smoothness. Let's take a smooth atlas $\lbrace (U_i, \varphi_i) \rbrace$ on $S^n$. Its transition maps are the following:
\[
\varphi_{i,j} = \varphi_i\circ\varphi_j^{-1}.
\]
Now, let's inspect the transition maps of the same atlas mapped back to $SO(n+1)/G$:
\[
\phi_{i,j} = \phi_i\circ\phi_j^{-1} = \varphi_i\circ\psi\circ\psi^{-1}\circ\varphi_j^{-1} = \varphi_i\circ\varphi_j^{-1}.
\]
We observe that the transition maps of both atlases are in fact the same and differentiable, in particular is $SO(n+1)/G$ a smooth manifold. Let's prove that $\psi$ is not merely a homeomorphism but a diffeomorphism as well.
Consider the following diagram:

\adjustbox{scale=1, center}{
	\begin{tikzcd}
	SO(n+1)/G \arrow[d, "\varphi_i\circ\psi"] \arrow[r, "\psi"] & S^n \arrow[d, "\varphi_j"] \\
	\R^n \arrow[r, "\Phi"] & \R^n
	\end{tikzcd}
}
For every two chart maps $\varphi_i$ and $\varphi_j$ we have that
\[
\Phi = \varphi_j\circ\psi\circ\psi^{-1}\circ\varphi_i^{-1} = \varphi_j\varphi_i^{-1},
\]
which is a transition chart of $S^n$, and hence differentiable. Same obviously holds for the inverse (switch $j$ and $i$). Since the chart maps were arbitrary and $\psi$ is a bijective map, $\psi$ is a diffeomorphism.
%If we take $\widetilde{\Phi} = id_{\R^n}$, we by definition get that $\Phi = \varphi_j^{-1}\circ\varphi_i\circ\psi$ is a diffeomorphism. In particular, for $i = j$, we get that $\psi$ is a diffeomorphism on every chart neighbourhood.
%Even more, taking $\widetilde{\Phi} = \varphi_j\circ\varphi_i^{-1}$ (the transition maps are differentiable and their inverses are transition maps), we see that $\phi$ is a diffeomorphism on all intersections of charts as well.
%Hence, $\psi$ is a diffeomorphism everywhere and $SO(n+1)/G$ is diffeomorphic to $S^n$.

Comment: We now see that we didn't need to check $2^{\text{nd}}$-countability and Hausdorffness, since we proved this by finding a homeomorphism.
\end{enumerate}

\underline{\textbf{Ex. 4:}}
Let $M$ be a smooth manifold of dimension $n$. We want to prove that its tangent bundle $TM$ is as orientable manifold.
Since we already know that $TM$ is a manifold (later we offer generic charts), we will concentrate on oriantability. Let $\mathcal{U} = \lbrace (U_\lambda, \varphi_\lambda) \rbrace$ be a smooth atlas for $M$. Then $T\mathcal{U} = \lbrace (TU_\lambda, T\varphi_\lambda) \rbrace$ is an atlas for $TM$. We are primarily interested in transition maps.
Take two charts $(U, \varphi), (V, \psi)$ where $U \cap V \neq \emptyset$. Then, by the functorial properties of $T$, we have
\[
T\varphi\circ (T\psi)^{-1} = T(\varphi\circ\psi) \colon \psi(U\cap V) \times \R^n \to \varphi(U \cap V) \times \R^n.
\]
Here is $\varphi\circ\psi^{-1}$ a differentiable transition map on $X$. Explicitly, we have
\[
T(\varphi\circ\psi^{-1})(x, v) = (\varphi\circ\psi^{-1}(x), D(\varphi\circ\psi^{-1})(x)v),
\]
where $x, v \in \R^n$ and $D(\varphi\circ\psi^{-1})(x)$ is the Jacobian of the transition map $\varphi\circ\psi^{-1}$ calculated at $x$.
Now we simply calculate the Jacobian of the tangent transition maps by components $(x, v) = (x_1, \dots, x_n, v_1, \dots, v_n)$. Since the first $n$ components of $T(\varphi\circ\psi^{-1})(x, v)$ are independent of $v$ and the last $n$ components are linear in $v$, we get by definition of $D(\varphi\circ\psi^{-1})(x)$ the following (simplified into $n\times n$ blocks):
\[
D(T(\varphi\circ\psi^{-1}))(x, v) =
\begin{bmatrix}
D(\varphi\circ\psi^{-1})(x) & 0 \\
D(\varphi\circ\psi^{-1})(x)_x & D(\varphi\circ\psi^{-1})(x)
\end{bmatrix}
\]
Since $\det D(\varphi\circ\psi^{-1})(x) \neq 0$, the Jacobian determinant (of a lower-triangular matrix) is now
\[
\det D(T(\varphi\circ\psi^{-1}))(x, v) = \det D(\varphi\circ\psi^{-1})(x) \cdot \det D(\varphi\circ\psi^{-1})(x) = (\det D(\varphi\circ\psi^{-1})(x))^2 > 0
\]
Here, $(U, \varphi)$ and $(V, \psi)$ are arbitrary charts, so for every transition map its Jacobian determinant is greater than $0$. By definition, $TM$ is an orientable manifold, since we have just found an oriented atlas.
\end{document}