\documentclass[a4paper, 12pt]{article}

\usepackage[slovene]{babel}
\usepackage[utf8]{inputenc}
\usepackage[T1]{fontenc}
\usepackage{lmodern}
\usepackage{units}
\usepackage{eurosym}
\usepackage{amsmath}
\usepackage{amssymb}
\usepackage{amsthm}
\usepackage{amsfonts}
\usepackage{mathtools}
\usepackage{graphicx}
\usepackage{color}
%\usepackage{url}
\usepackage{hyperref}
\usepackage{enumerate}
\usepackage{enumitem}
\usepackage{pifont}
\usepackage{tikz-cd}
\usetikzlibrary{babel}
\usepackage{adjustbox}

% set margin and layout here
\usepackage[margin=0.5in]{geometry}

% commonly used math operators
\DeclareMathOperator{\diam}{diam}
\DeclareMathOperator{\rank}{rank}
\DeclareMathOperator{\im}{im}
\DeclareMathOperator{\coker}{coker}
\DeclareMathOperator{\Lin}{Lin}
\DeclareMathOperator{\Ann}{Ann}
\DeclareMathOperator{\Ass}{Ass}
\DeclareMathOperator{\Spec}{Spec}
\DeclareMathOperator{\mSpec}{mSpec}
\DeclareMathOperator{\Quot}{Quot}
\DeclareMathOperator{\Tor}{Tor}
\DeclareMathOperator{\Ext}{Ext}
\DeclareMathOperator{\Hom}{Hom}
\DeclareMathOperator{\Hol}{Hol}

% commonly used math objects
\newcommand{\D}{\mathbb{D}}
\renewcommand{\S}{\mathbb{S}}
\newcommand{\B}{\mathbb{B}}
\newcommand{\N}{\mathbb{N}}
\newcommand{\Z}{\mathbb{Z}}
\newcommand{\Q}{\mathbb{Q}}
\newcommand{\R}{\mathbb{R}}
\newcommand{\C}{\mathbb{C}}
\renewcommand{\P}{\mathbb{P}}
\newcommand{\F}{\mathbb{F}}

% commonly used math relations
\newcommand{\iso}{\cong}
\newcommand{\homeo}{\approx}
\newcommand{\htpeq}{\simeq}
\newcommand{\hlgeq}{\sim}
\newcommand{\idtfy}{\longleftrightarrow}

% commonly used math symbols
\newcommand{\closure}[1]{\overline{#1}}
\newcommand{\subideal}{\vartriangleleft}
\newcommand{\supideal}{\vartriangleright}

% title data - MODIFY
\title{Funkcionalna analiza - $2.$ domača naloga}
\author{Benjamin Benčina, 27192018}

\begin{document}

\maketitle

\underline{\textbf{Nal. 1:}}
Naj bo $T\colon X \to Y$ izometrični izomorfizem normiranih prostorov.
\begin{enumerate}[label=(\alph*)]
	\item Dokažimo, da $T$ preslika ekstremne točke zaprte enotske krogle $\B_X$ prostora $X$ v ekstremne točke zaprte enotske krogle $\B_Y$ prostora $Y$. Za dokaz s protislovjem vzemimo ekstremno točko $a \in \Ext\B_X$ in predpostavimo, da $Ta \notin \Ext\B_Y$ (seveda $Ta \in \B_Y$, saj je $T$ izometrični izomorfozem). Potem obstajata točki $y,z \in \B_y$ in $\lambda \in (0,1)$, da je $Ta = \lambda y + (1 - \lambda) z$ in $Ta \neq y, z$. Vendar pa je $T$ kot izomorfizem obrnljiva linearna preslikava, zato enačbo komponiramo s $T^{-1}$ in dobimo $a = \lambda T^{-1}y + (1 - \lambda) T^{-1}z$. Ker pa je $a$ ekstremna točka, mora veljati $a = T^{-1}y = T^{-1}z$, kar je v protislovju z bijektivnostjo preslikave $T$.
	\item Dokažimo, da je preslikava $T\colon c \to c_0$, definirana s predpisom
	\[
	T\colon (x_1, x_2, \dots) \mapsto (\lim_{n\to\infty} x_n, \lim_{n\to\infty} x_n - x_1, \lim_{n\to\infty} x_n - x_2, \dots),
	\]
	dobro definirani topološki izomorfizem Banachovih prostorov.
	\begin{itemize}
		\item Vemo že, da sta $c$ in $c_0$ Banachova prostora glede na maksimum normo ($c$ je zaprt v $l^\infty$ in $c_0$ je zaprt v $c$).
		\item Linearnost je očitna, saj je limitni operator linearen, preslikava pa je definirana po komponentah.
		\item Bijektivnost je prav tako očitna, saj vidimo, da slika elementa z operatorjem $T$ nosi vse informacije o elementu. Konkretno imamo inverz
		\[
		(a, a_1, a_2, \dots) \mapsto (a - a_1, a - a_2, \dots).
		\]
		\item Operator $T$ je dobro definiran, saj
		\[
		\lim_{}\lbrace x, x - x_1, x - x_2, \dots \rbrace = \lim_{}\lbrace x - x_1, x - x_2, \dots \rbrace = x - x = 0,
		\]
		kjer je $x = \lim_{n\to\infty} x_n$.
		\item Oglejmo si zveznost preslikave $T$. Zopet označimo $x = \lim_{n\to\infty} x_n$ in računajmo
		\begin{align*}
		||T(x_n)_n||_{\infty} &= || (x, x - x_1, x - x_2 \dots) ||_\infty = \max\lbrace |x|, |x - x_1|, |x - x_2|, \dots \rbrace \\
		&\leq \max\lbrace |x|, |x| + |x_1|, |x| + |x_2|, \dots \rbrace = \max\lbrace |x| + |x_n| ;\; n \in \N\rbrace \\
		&= |x| + ||(x_n)_n||_\infty \leq 2 ||(x_n)_n||_\infty
		\end{align*}
		Tukaj lahko $T$ podobno kot v prejšnji domači nalogi razumemo kot neke vrste identifikacijo. Po posledici izreka o odprti preslikavi je $T$ topološki izomorfizem.
	\end{itemize}
	\item Ali sta Banachova prostora $c$ in $c_0$ izometrično izomorfna? Odgovor je ne, dokazali bomo s pomočjo točke (a). Zaporedje $x = (1, 1, \cdots)$ je jasno ekstremna točka $\B_c$, vendar pa je $Tx = (1, 0, 0,\cdots)$. Zlahka najdemo zapis
	\[
	(1, 0, 0, \cdots) = \frac{1}{2}(1, \frac{1}{2}, \cdots, \frac{1}{2^n}, \cdots) + \frac{1}{2}(1, -\frac{1}{2}, \cdots, -\frac{1}{2^n}, \cdots),
	\]
	torej $(1, 0, 0, \cdots)$ ni ekstremna točka operatorja $T$. Po točki (a) operator $T$ ne more biti izometrija.
\end{enumerate}

\underline{\textbf{Nal. 2:}}
Naj bo $K$ kompaktna podmnožica normiranega prostora $X$ in naj bo $(f_n)_n$ omejeno zaporedje omejenih funkcionalov na $X$.
\begin{enumerate}[label=(\alph*)]
	\item Dokažimo, da obstaja tako podzaporedje $(f_{n_k})_k$, ki konvergira enakomerno na $K$. Ker je $X$ normiran prostor, je Hausdorffov, zato je Hausdorffov tudi podprostor $K$. Seveda je $\F$ metričen prostor, zato so izpolnjeni predpogoji za izrek Arzel\`a-Ascoli (A-A). Označimo $\mathcal{F} = \lbrace f_n|_K ;\; n \in \N \rbrace$.
	\begin{itemize}
		\item Ker je $\mathcal{F}$ družina zveznih omejenih funkcij (na kompaktni množici), je $\mathcal{F}$ enakoomejena natanko tedaj, ko je omejena.
		\item Preverjamo, da za vsak $x \in K$ in $\epsilon > 0$ obstaja okolica $U_x \subset K$ elementa $x$, da za vsak $f \in \mathcal{F}$ in $x' \in U_x$ velja $| f(x) - f(x') | < \epsilon$. Fiksirajmo $\epsilon > 0$.
		Upoštevamo linearnost in omejenost funkcionalov
		\[
		|f(x) - f(x')| = |f(x - x')| \leq ||f||\cdot |x-x'| < \epsilon
		\]
		Ker je $\mathcal{F}$ omejena družina (naj bo omejena z $M$), je za $U_x$ dovolj vzeti odprte krogle polmera manj kot $\frac{\epsilon}{M}$.
		%Preverjamo, da za vsak $x \in K$ in $\epsilon > 0$ obstaja okolica $U_x \subset K$ elementa $x$, da za vsak $f \in \mathcal{F}$ in $x' \in U_x$ velja $| f(x) - f(x') | < \epsilon$. Fiksirajmo $\epsilon > 0$. Ker so vsi funkcionali iz $\mathcal{F}$ zvezni, za vsak $f \in \mathcal{F}$ obstaja $\delta_f$, da za vsak $x \in K$ velja $|x - x'| < \delta_f \implies |f(x) - f(x')| < \epsilon$. Te
	\end{itemize}
	Sedaj lahko uporabimo A-A in dobimo podzaporedje $\lbrace f_{n_k}|_K\rbrace_k$, ki konvergira enakomerno na $K$. Iskano podzaporedje je torej $\lbrace f_{n_k}\rbrace_k$.
	\item Naj bo $K$ kompaktna podmnožica v Banachovem prostoru $(C^1[0,1], ||.||_1)$. Dokažimo, da za vsako zaporedje $(x_n)_n$ v $[0,1]$ obstaja tako podzaporedje $(x_{n_k})_k$, da za vsak $\epsilon > 0$ obstaja tak $k_0 \in \N$, da za vse indekse $k, l > k_0$ velja $|f'(x_{n_k}) - f'(x_{n_l})| < \epsilon$ za vse $f \in K$.
	
	Najprej si oglejmo nekaj dejstev, ki sledijo iz navodil. Ker so vse funkcije, ki jih obravnavamo, definirane na kompaktnem intervalu $[0, 1]$, je vsaka posebej omejena. Prostor $C^1[0, 1]$ je poln (Banachov) in metričen, torej povsem omejen (totally bounded). Kompaktna družina funkcij $\mathcal{F} = \lbrace f \in C^1[0, 1] ;\; f \in K\rbrace$ je torej omejena (naj bo omejena s konstanto $M$). Spomnimo se, da je operator odvoda prav tako omejen (naj bo omejen s konstanto $N$). Družina $\mathcal{F}' = \lbrace f' ; \; f \in \mathcal{F} \rbrace$ je zato omejena (s konstanto $NM$). Vzemimo sedaj poljubno zaporedje $(x_n)_n \subset [0, 1]$ in fiksirajmo $\epsilon > 0$. Ker je $[0, 1]$ kompaktna množica (in poln prostor glede na običajno normo), obstaja tako podzaporedje $(x_{n_k})_k$ in indeks $k_0$, da je $|x_{n_k} - x_{n_l}| < \epsilon$ za vse $k, l > k_0$. Potem pa po zveznosti funkcij $f'$ in omejenosti družine $\mathcal{F}'$ sledi $|f'(x_{n_k}) - f'(x_{n_l})| < 2 N M \epsilon$ za vsaka $k, l > k_0$ in $f' \in \mathcal{F}'$. Na koncu le prilagodimo začetni $\epsilon$, da se znebimo konstante.
\end{enumerate}

\underline{\textbf{Nal. 3:}}
Naj bo $T$ tak omejen operator na kompleksnem Banachovem prostoru $X$, da je $\sigma(T) = E \cup F$ za neki neprazni disjunktni zaprti podmnožici spektra $\sigma(T)$. Z drugimi besedami, spekter operatorja $T$ sestavljata dve komponenti za povezanost.
\begin{enumerate}[label=(\alph*)]
	\item Najprej utemeljimo, da je funkcija $f = \chi_E \in \Hol(T)$. To je jasno, saj je $f$ konstantna funkcija, definirana na dveh disjunktnih zaprtih podmnožicah kompleksne ravnine (na komponentah ima lahko različne vrednosti), in je kot taka holomorfna.
	\item Dokažimo, da je $f(T)$ idempotenten operator, ki komutira z operatorjem $T$. Naj bo $K$ kompakt, ki vsebuje $E \cup F$. Brez škode za splošnost lahko privzamemo, da je $K$ sestavljen iz dveh disjunktnih kompaktov $K_1$ in $K_2$, vsak pa zaporedoma vsebuje eno od množic $E$ in $F$. Računamo
	\begin{align*}
	\chi_E(T) &= \frac{1}{2\pi i} \int_{\partial K} (zI - T)^{-1}\chi_E(z) dz = \frac{1}{2\pi i}\left(\int_{\partial K_1} (zI - T)^{-1}\chi_E(z) dz + \int_{\partial K_2} (zI - T)^{-1}\chi_E(z) dz \right)\\
	&= 1_E(T) + 0 =
	\begin{bmatrix}
	I & 0 \\
	0& 0
	\end{bmatrix}
	\end{align*}
	kjer upoštevamo, da v primeru konstantne funkcije $1$ dobimo enotski operator, v primeru ničelne funkcije dobimo ničelni operator (oboje izrek s predavanj) in da je rezultat neodvisen od izbire množice $K$. Obe željeni lastnosti sledita iz oblike operatorja.
	\item Dokažimo še, da obstajata taka zaprta podprostora $Y$ in $Z$ prostora $X$, da je $X = Y \oplus Z$, ki sta invariantna za operator $T$. Spomnimo se vaj in vzemimo $Y = \ker f(T)$ ter $Z = \im f(T)$. Preverimo invariantnost:
	\begin{itemize}
		\item Naj bo $u \in \ker f(T)$. Potem je
		\[
		f(T)(Tu) = T(f(T)u) = T(0) = 0,
		\]
		torej je $Tu \in \ker f(T)$.
		\item Naj bo $u \in \im f(T)$. Potem obstaja $v \in X$, da je $f(T)v = u$. Upoštevamo idempotentnost in dobimo
		\[
		f(T)v = f(T)(f(T)v) = f(T)u.
		\]
		Računamo
		\[
		Tu = T(f(T)v) ) = T(f(T)u) = f(T)(Tu),
		\]
		torej je $Tu \in \im f(T)$.
	\end{itemize}
\end{enumerate}

\underline{\textbf{Nal. 4:}}
Prostor $n$-krat zvezno odvedljivih funkcij $C^n[a,b]$ opremimo z normo $|| f || = \Sigma_{k=0}^n \frac{1}{k!}|| f^{(k)} ||_\infty$, da postane komutativna Banachova algebra z enoto.
\begin{enumerate}[label=(\alph*)]
	\item Najprej pokažimo, da je zgornja norma res submultiplikativna. Računamo
	\begin{align*}
	|| fg || &= \Sigma_{k=0}^n \frac{1}{k!} || (fg)^{(k)} ||_\infty = \Sigma_{k=0}^n \frac{1}{k!} || \Sigma_{i=0}^k \binom{k}{i} f^{(k-i)} g^{(i)} ||_\infty \\
	&\leq \Sigma_{k=0}^n \frac{1}{k!} \Sigma_{i=0}^k \binom{k}{i} || f^{(k-i)}||_\infty || g^{(i)} ||_\infty = \Sigma_{k=0}^n \Sigma_{i=0}^k \frac{1}{(k-i)!} || f^{(k-i)}||_\infty \frac{1}{i!} || g^{(i)} ||_\infty \\
	&\leq \Sigma_{k=0}^n \Sigma_{l=0}^n \frac{1}{(k)!} || f^{(k)}||_\infty \frac{1}{l!} || g^{(l)} ||_\infty = || f || \cdot || g ||,
	\end{align*}
	kjer je zadnja neenakost upravičena s tem, da na levi strani manjkajo členi produkta vsot.
	\item Dokažimo, da je za vsak $x \in [a, b]$ množica $J_x = {f \in C^n[a, b] ;\; f(x) = 0}$ maksimalni ideal v $C^n[a, b]$. Jasno je, da je ideal, saj sta produkt in vsota funkcij definirana po točkah. Za dokaz maksimalnosti s protislovjem privzemimo, da obstaja ideal $M \subideal C^n[a, b]$, da je $J_x \subset M$. Potem obstaja funkcija $g \in M$, da $g(x) \neq 0$, saj bi sicer bila že v $J_x$. Zato lahko definiramo funkciji $h(y) = \frac{g(y)}{g(x)} \in M$ in $(1 - h(y)) \in J_x$, saj je $1 - h(x) = 1 - 1 = 0$. Vendar pa sedaj velja
	\[
	1 = h(y) + (1 - h(y)) \in M \implies M = C^n[a, b].
	\]
	Po definiciji je $J_x$ maksimalni ideal.
	\item Dokažimo še obratno, da so vsi maksimalni ideali v $C^n[a, b]$ oblike $J_x$ za nek $x \in [a, b]$.
	Za dokaz s protislovjem privzemimo, da je $M \in \mSpec C^n[a, b]$, ki ni oblike $J_x$ za nek $x \in [a, b]$. Potem za vsak $x \in [a, b]$ obstaja funkcija $f_x \in M$, da $f_x(x) \neq 0$. Ker je $f_x$ zvezna (celo $n$-krat zvezno odvedljiva), obstaja odprta okolica $U_x$ točke $x$, da $f_x(t) \neq 0$ za vse $t \in U_x$. Dobili smo odprto pokritje $\lbrace U_x ;\; x \in [a, b] \rbrace$ za kompaktno množico $[a, b]$, zato ostajajo $x_1, \dots, x_n$, da $U_i = U_{x_i}$, kjer $i = 1, \dots, n$ pokrijejo $[a, b]$. S $f_i$ za $i = 1, \dots, n$ označimo pripadajoče funkcije iz zgornjega nabora, torej $f_i(t) \neq 0$ za vsak $t \in U_i$. Končno definiramo
	\[
	g = f_1^2 + \cdots f_n^2 \in M
	\]
	Seveda velja $g(t) > 0$ za vsak $t \in [a, b]$, zato je $g$ obrnljiva funkcija v kolobarju $C^n[a, b]$. Sledi, da je $M = C^n[a, b]$.
	\item Za konec dokažimo še, da je $\mathcal{A} = (C^n[a, b], ||.||)$ polenostavna Banachova algebra, njena Gelfandova reprezentacija pa ni ne izometrična ne surjektivna.
	\begin{itemize}
		\item Iz (c) sledi, da so v $\mathcal{A}$ vsi maksimalni ideali oblike $J_x$, njihovi pripadajoči karakterji pa so $\varphi_x(f) = f(x)$. Res, $J_x = \ker\varphi_x$. Sledi, da je za vsak $x \in [a, b]$ preslikava $\varphi_x$ karakter na $\mathcal{A}$. Po definiciji je potem $\sigma(f) = \lbrace \varphi_x(f) ;\; \varphi_x \text{ karakter}\rbrace = \lbrace f(x) ;\; x \in [a, b] \rbrace = \im f$. Preverimo injektivnost reprezentacije. Če je $\varphi_x(f) = 0$ za vse $x \in [a, b]$, potem je $f(x) = 0$ za vse $x \in [a, b]$, torej $f \equiv 0$. Sledi, da je $\mathcal{A}$ polenostavna.
		\item Iz primera s predavanj\footnote{Predavanja 24. maja, primer 1.} sledi, da je $\mathcal{A} \homeo [a, b]$. Računamo
		\[
		\gamma(f)(\varphi_x) = \overset{\wedge}{f}(\varphi_x) = \varphi_x(f) = f(x),
		\]
		kjer je $x$ povsod isti element. Glede na identifikacijo $\varphi_x \idtfy x$ je preslikava $\gamma$ torej identična. Če zaporedoma identificiramo domeno in kodomeno, dobimo
		\[
		\gamma\colon \mathcal{A} \to \overset{\wedge}{\mathcal{A}} \idtfy \gamma \colon C^n[a,b] \to C[a,b]
		\]
		Ker je $C^n[a, b] \subset C[a, b]$ strogo, $\gamma$ ne more biti surjektivna.
		\item Velja $|| \gamma(f) ||_\infty = || f || \iff || f ||_\infty = || f ||$ za vsak $f \in \mathcal{A}$. Desna stran seveda ni resnična, saj je funkcija $f(x) = x$ protiprimer. Res,
		\begin{align*}
		& || f ||_\infty = b \\
		& || f || = 1 \cdot b + 1 \cdot 1 + 0 + \cdots + 0 = b + 1
		\end{align*}
	\end{itemize}
\end{enumerate}

\end{document}

%% TEMPLATES
% lists
%\begin{enumerate}[label=(\alph*)]
% diagram
%\adjustbox{scale=1, center}{
%	\begin{tikzcd}
%		\R_n \arrow[d, "\varphi_n"] \arrow[r, "\Phi"] & \R_m \arrow[d, "\varphi_m"] \\
%		\R \arrow[r, "\widetilde{\Phi}"] & \R
%	\end{tikzcd}
%}
% figure
%\begin{figure}[h]
%	\centering
%	\includegraphics[scale=0.4]{fig}
%	\caption{caption}
%	\label{fig:label}
%\end{figure}