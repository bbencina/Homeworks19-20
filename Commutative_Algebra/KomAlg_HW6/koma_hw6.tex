\documentclass[a4paper, 12pt]{article}

\usepackage[slovene]{babel}
\usepackage[utf8]{inputenc}
\usepackage[T1]{fontenc}
\usepackage{lmodern}
\usepackage{units}
\usepackage{eurosym}
\usepackage{amsmath}
\usepackage{amssymb}
\usepackage{amsthm}
\usepackage{amsfonts}
\usepackage{mathtools}
\usepackage{graphicx}
\usepackage{color}
%\usepackage{url}
\usepackage{hyperref}
\usepackage{enumerate}
\usepackage{enumitem}
\usepackage{pifont}

% set margin and layout here
\usepackage[margin=0.5in]{geometry}

% commonly used math operators
\DeclareMathOperator{\diam}{diam}
\DeclareMathOperator{\rank}{rank}
\DeclareMathOperator{\im}{im}
\DeclareMathOperator{\Lin}{Lin}
\DeclareMathOperator{\Ann}{Ann}
\DeclareMathOperator{\Spec}{Spec}
\DeclareMathOperator{\mSpec}{mSpec}

% commonly used math objects
\newcommand{\D}{\mathbb{D}}
\renewcommand{\S}{\mathbb{S}}
\newcommand{\B}{\mathbb{B}}
\newcommand{\N}{\mathbb{N}}
\newcommand{\Z}{\mathbb{Z}}
\newcommand{\Q}{\mathbb{Q}}
\newcommand{\R}{\mathbb{R}}
\newcommand{\C}{\mathbb{C}}
\renewcommand{\P}{\mathbb{P}}

% commonly used math symbols
\newcommand{\closure}[1]{\overline{#1}}
\newcommand{\subideal}{\vartriangleleft}
\newcommand{\supideal}{\vartriangleright}

% title data - MODIFY
\title{Komutativna algebra - $6.$ domača naloga}
\author{Benjamin Benčina, 27192018}

\begin{document}

\maketitle

\underline{\textbf{Nal. 1:}}
Naj bo $M$ $R$-modul in $\lbrace m_\lambda ; \; \lambda \in \Lambda\rbrace \subset M$ neka podmnožica.
Pokažimo, da ta množica generira modul $M$ natanko tedaj, ko njena slika $\lbrace \frac{m_\lambda}{1} ;\; \lambda \in \Lambda\rbrace$ generira $R_\mathbf{m}$-modul $M_\mathbf{m}$ za vsak $\mathbf{m} \in \mSpec R$. 

Najprej privzemimo $m = \Sigma_{i=1}^{n}r_im_i$ in računajmo
\[
\frac{m}{s} = \frac{\Sigma_{i=1}^{n}r_im_i}{s} = \frac{s^{n-1}}{s^{n-1}} \frac{\Sigma_{i=1}^{n}r_im_i}{s} = \frac{\Sigma_{i=1}^{n}s^{n-1}r_im_i}{s^n} = \Sigma_{i=1}^n \frac{r_im_i}{s} = \Sigma_{i=1}^n \frac{r_i}{s}\frac{m_i}{1}.
\]
Ta stranski račun nam je že pokazal implikacijo iz leve v desno.
Sedaj pokažimo ekvivalenco. Modul, generiran z množico $A$, bo označen z $\Lin A$.

Ker je $\Lin\lbrace m_\lambda ;\;\lambda \in \Lambda\rbrace \leq M$, bo enakost veljala natanko tedaj, ko bo kvocient $M/\Lin\lbrace m_\lambda ;\;\lambda \in \Lambda\rbrace = 0$. Po trditvi 5.16 iz predavanj, je to res natanko tedaj, ko za vsak $\mathbf{m} \in \mSpec R$ velja $\left(M/\Lin\lbrace m_\lambda ;\;\lambda \in \Lambda\rbrace\right)_\mathbf{m} = 0$. Seveda pa je lokalizacija kvocienta kar kvocient lokalizacij, zato bo to veljalo natanko tedaj, ko bo $M_\mathbf{m} = (\Lin\lbrace m_\lambda ;\;\lambda \in \Lambda\rbrace)_\mathbf{m}$. Množica na desni, je po prejšnjem stranskem računu enaka $\Lin\lbrace \frac{m_\lambda}{1} ;\; \lambda \in \Lambda\rbrace$ (stranski račun pokaže vsebovanost v desno, druga vsebovanost je trivialna). S tem je trditev dokazana.

\underline{\textbf{Nal. 2:}}
Naj bo $M$ Noetherski $R$-modul. Pokažimo, ekvivalence.

Privzemimo (a). Ker je $M$ Noetherski s končno dolžino, je po trditvi 6.7 Artinski. Modul $R/\Ann M$ je njegov podmodul, zato je po eksaktnosti tudi Artinski. S tem smo dokazali točko (d). Od tukaj nadaljujemo. Ker je $R/\Ann M$ Artiniski, po trditvi 6.11a iz predavanj obstajajo $P_1,\dots,P_n \in \mSpec (R/\Ann M)$, da je produkt $P_1\cdots P_n = (0)$. Ko naredimo kontrakcijo s kvocientno preslikavo $q\colon R \to R/\Ann M$, dobimo končen produkt maksimalnih idealov $q^{-1}(P_1)\cdot q^{-1}(P_n) \subseteq \Ann M$. S tem smo dokazali točko (b). Sedaj smo spomnimo, da smo pri dokazu trditve 6.11b zares potrebovali le trditev 6.11a Torej iz (b) sledi, da $\Spec (R/\Ann M) = \mSpec (R/\Ann M)$. Kontrakcija s kvocientno preslikavo $q$ dokaže še točko (c).
(Trenutno shema je (a) $\implies$ (d) $\implies$ (b) $\implies$ (c)). Ker je $R/\Ann M$ Noetherski kot podmodul v $M$, po 6.12 (Hopkinsov izrek) iz (c) sledi (d).
(Trenutno shema je (a) $\implies$ (d) $\iff$ (b) $\iff$ (c))

 Dokazati moramo le še, da iz katerekoli izmed drugih točk sledi (a). Najlažje bo, če privzamemo (d). Ker je $M$ Noetherski, je po trditvi 6.3c tudi končno generiran $R$-modul, jasno pa ga lahko ekvivalentno vidimo kot končno generiran $R/\Ann M$ (saj smo le faktorizirali elemente, ki uničijo $M$). Potem pa je $M$ nek kvocient modula $\left( R/\Ann M\right)^n$ za nek $n \in \N$, ki pa je Artinski, saj je Artinski tudi modul $\left(R/\Ann M\right)^n$ kot končna vsota po predpostavki Artinskih modulov (trditev 6.6a).
 Modul $M$ je torej Artinski in po predpostavki Noetherski, zato ima po trditvi 6.7 končno dolžino. To zaključi dokaz ekvivalenc.

\end{document}

%% TEMPLATES
%\begin{enumerate}[label=(\alph*)]
