\documentclass[a4paper, 12pt]{article}

\usepackage[slovene]{babel}
\usepackage[utf8]{inputenc}
\usepackage[T1]{fontenc}
\usepackage{lmodern}
\usepackage{units}
\usepackage{eurosym}
\usepackage{amsmath}
\usepackage{amssymb}
\usepackage{amsthm}
\usepackage{amsfonts}
\usepackage{mathtools}
\usepackage{graphicx}
\usepackage{color}
%\usepackage{url}
\usepackage{hyperref}
\usepackage{enumerate}
\usepackage{enumitem}
\usepackage{pifont}
\usepackage{tikz-cd}
\usetikzlibrary{babel}
\usepackage{adjustbox}

% set margin and layout here
\usepackage[margin=0.5in]{geometry}

% commonly used math operators
\DeclareMathOperator{\diam}{diam}
\DeclareMathOperator{\rank}{rank}
\DeclareMathOperator{\im}{im}
\DeclareMathOperator{\coker}{coker}
\DeclareMathOperator{\Lin}{Lin}
\DeclareMathOperator{\Ann}{Ann}
\DeclareMathOperator{\Ass}{Ass}
\DeclareMathOperator{\Spec}{Spec}
\DeclareMathOperator{\mSpec}{mSpec}
\DeclareMathOperator{\Quot}{Quot}
\DeclareMathOperator{\Tor}{Tor}
\DeclareMathOperator{\Ext}{Ext}

% commonly used math objects
\newcommand{\D}{\mathbb{D}}
\renewcommand{\S}{\mathbb{S}}
\newcommand{\B}{\mathbb{B}}
\newcommand{\N}{\mathbb{N}}
\newcommand{\Z}{\mathbb{Z}}
\newcommand{\Q}{\mathbb{Q}}
\newcommand{\R}{\mathbb{R}}
\newcommand{\C}{\mathbb{C}}
\renewcommand{\P}{\mathbb{P}}

% commonly used math relations
\newcommand{\iso}{\cong}
\newcommand{\homeo}{\approx}
\newcommand{\htpeq}{\simeq}
\newcommand{\hlgeq}{\sim}
\newcommand{\idtfy}{\longleftrightarrow}

% commonly used math symbols
\newcommand{\closure}[1]{\overline{#1}}
\newcommand{\subideal}{\vartriangleleft}
\newcommand{\supideal}{\vartriangleright}

% title data - MODIFY
\title{Komutativna algebra - $10.$ domača naloga}
\author{Benjamin Benčina, 27192018}

\begin{document}

\maketitle

\underline{\textbf{Nal. 1:}}
Pokažimo, da je $R \subseteq R'$ celostna razširitev natanko tedaj, ko je $R[x] \subseteq R'[x]$ celostna razširitev.

Za implikacijo iz leve v desno predpostavimo, da je $R'$ celosten nad $R$. Potem je $R'$ celosten tudi nad $R[x]$ na naraven način. Prav tako je polinom $x \in R[x]$ celosten nad $R[x]$ trivialno. Ker so vsote in produkti celostnih elementov spet celostni (spomnimo se, da vsi celostni elementi tvorijo kolobar) in ker so vsi elementi iz $R'[x]$ $R'$-linearne kombinacije polinoma $x$ in njegovih potenc (vštevši ničelno potenco), implikacija sledi.

Za obratno implikacijo privzemimo, da je $R'[x]$ celosten nad $R[x]$, in vzemimo $a \in R'$. Iščemo moničen polinom iz $R[x]$, ki uniči $a$. Ker je $a \in R' \subseteq R'[x]$, obstaja moničen polinom $p \in R[x][y]$, ki uniči $a$. Ta polinom (dveh spremenljivk) združimo po potencah $x$ in dobimo
\[
p_0(y) + x p_1(y) + \cdots + x^np_n(y),
\]
ki je še vedno isti polinom, ki uniči $a$ (torej, če vstavimo $y = a$, dobimo $0$). Vendar, če vstavimo $y = a$, dobimo polinomsko enačbo oblike $P(x) = 0$. Ker je polinom ničelen natanko tedaj, ko ima ničelne vse koeficiente, ima $P(x)$ ničelen konkretno tudi konstantni člen. Torej že polinom $p_0(y) \in R[y] \iso R[x]$ uniči $a$ in tudi druga implikacija sledi.
\newline

\underline{\textbf{Nal. 2:}}
Razširitev $R \subseteq R'$ zadošča lastnosti gor grede, če zanjo zadošča zaključek gor grede izreka (9.16), torej če za vsaka dva praideala $P \in \Spec R$ in $P' \in \Spec R'$, kjer $P = P' \cap R$, ter praideal $Q \in \Spec R$, kjer $P \subseteq Q$, obstaja praideal $Q' \in \Spec R'$, da je $P' \subseteq Q'$ in $Q = Q' \cap R$.
Pokažimo, da razširitev $R \subseteq R'$ zadošča lastnosti gor grede natanko tedaj, ko je za vsaka dva praideala $P \in \Spec R$ in $P' \in \Spec R'$, kjer $P = P' \cap R$, naravna preslikava $\varphi \colon \Spec (R'/P') \to \Spec (R/P)$ surjektivna.

Podrobneje si oglejmo naravno preslikavo $\varphi$. Ker je $R\subseteq R'$, je $\varphi$ definirana s predpisom
\[
Q' + P' \mapsto (Q' + P') \cap R = \underbrace{Q' \cap R}_{Q \subideal R} + \underbrace{P' \cap R}_P,
\]
kjer je s $Q' + P'$ označen praideal $Q'/P'$.
Surjektivnost preslikave $\varphi$ torej pomeni, da za vsak praideal $Q + P$ obstaja praideal $Q' + P'$, da $\varphi(Q' + P') = Q + P$.
Tukaj se spomnimo, da so praideali v kvocientnem kolobarju $R/P$ točno praideali v $R$, ki vsebujejo $P$ (in analogno za $R'/P'$).
Surjektivnost preslikave $\varphi$ torej pomeni, da za vsak praideal $Q \supseteq P$ obstaja praideal $Q' \supseteq P'$, za katerega (po definiciji $\varphi$) velja $Q' \cap R = Q$, kar je očitno ekvivaletna izjava definiciji lastnosti gor grede.

\end{document}

%% TEMPLATES
% lists
%\begin{enumerate}[label=(\alph*)]
% diagram
%\adjustbox{scale=1, center}{
%	\begin{tikzcd}
%		\R_n \arrow[d, "\varphi_n"] \arrow[r, "\Phi"] & \R_m \arrow[d, "\varphi_m"] \\
%		\R \arrow[r, "\widetilde{\Phi}"] & \R
%	\end{tikzcd}
%}
% figure
%\begin{figure}[h]
%	\centering
%	\includegraphics[scale=0.4]{fig}
%	\caption{caption}
%	\label{fig:label}
%\end{figure}