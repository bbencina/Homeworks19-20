\documentclass[a4paper, 12pt]{article}

\usepackage[slovene]{babel}
\usepackage[utf8]{inputenc}
\usepackage[T1]{fontenc}
\usepackage{lmodern}
\usepackage{units}
\usepackage{eurosym}
\usepackage{amsmath}
\usepackage{amssymb}
\usepackage{amsthm}
\usepackage{amsfonts}
\usepackage{mathtools}
\usepackage{graphicx}
\usepackage{color}
%\usepackage{url}
\usepackage{hyperref}
\usepackage{enumerate}
\usepackage{enumitem}
\usepackage{pifont}
\usepackage{tikz-cd}
\usetikzlibrary{babel}
\usepackage{adjustbox}

% set margin and layout here
\usepackage[margin=0.5in]{geometry}

% commonly used math operators
\DeclareMathOperator{\diam}{diam}
\DeclareMathOperator{\rank}{rank}
\DeclareMathOperator{\im}{im}
\DeclareMathOperator{\Lin}{Lin}
\DeclareMathOperator{\Ann}{Ann}
\DeclareMathOperator{\Spec}{Spec}
\DeclareMathOperator{\mSpec}{mSpec}
\DeclareMathOperator{\Ass}{Ass}

% commonly used math objects
\newcommand{\D}{\mathbb{D}}
\renewcommand{\S}{\mathbb{S}}
\newcommand{\B}{\mathbb{B}}
\newcommand{\N}{\mathbb{N}}
\newcommand{\Z}{\mathbb{Z}}
\newcommand{\Q}{\mathbb{Q}}
\newcommand{\R}{\mathbb{R}}
\newcommand{\C}{\mathbb{C}}
\renewcommand{\P}{\mathbb{P}}

% commonly used math symbols
\newcommand{\closure}[1]{\overline{#1}}
\newcommand{\subideal}{\vartriangleleft}
\newcommand{\supideal}{\vartriangleright}

% title data - MODIFY
\title{Komutativna algebra - $7.$ domača naloga}
\author{Benjamin Benčina, 27192018}

\begin{document}

\maketitle

\underline{\textbf{Nal. 1:}}
Naj bo $R$ cel Noetherski kolobar.
\begin{enumerate}[label=(\alph*)]
	\item Pokažimo, da je $R$ kolobar z enolično faktorizacijo (UFD) natanko tedaj, ko so vsi minimalni praideali nad glavnimi ideali spet glavni ideali.
	
	Privzemimo najprej, da je $R$ UFD in vzemimo poljuben glavni ideal $(a)$. Ker je $R$ Noetherski kolobar, so po posledici 7.20 s predavanj minimalni praideali nad $(a)$ točno izolirani praideali ideala $(a)$, torej minimalni elementi množice $\Ass((a))$. Ker je $R$ Noetherski, ima vsak ideal minimalno primarno dekompozicijo, zato naj bo $\lbrace Q_1, \dots, Q_n\rbrace$ minimalna primarna dekompozicija za ideal $(a)$. Po izreku 7.19 velja $\Ass((a)) = \lbrace P_1, \dots, P_n \rbrace$, kjer je $P_i = \sqrt{Q_i}$.
	Ker je $R$ UFD, naj bo $a = p_1^{k_1}\cdots p_m^{k_m}$ do vrstnega reda in asociacije enoličen zapis elementa $a$ kot produkt praelementov. Ker je $a \in \bigcap_{i=1}^m Q_i \subseteq \bigcap_{i=1}^m P_i$, vsak od idealov $P_i$ vsebuje vsaj eno potenco nekega praelementa $p_j$. Ker smo dobili ideale $P_i$ iz minimalne dekompozicije in ker je radikal preseka enak preseku radikalov, je $n=m$ in (po preindeksiranju) zaradi minimalnosti praidealov velja $P_i = (p_i)$.
	
	Obratno, naj bo vsak minimalni praideal nad glavnim idealom tudi glavni. Uporabili bomo dejstvo (izrek Kaplanskega), da je cel kolobar UFD natanko tedaj, ko vsak praideal vsebuje praelement. Naj bo $Q$ nek praideal. Vzemimo poljuben neničelen element $a \in Q$ in si oglejmo glavni ideal $(a)$. Na isti način kot zgoraj dobimo (končno) množico minimalnih praidealov nad $(a)$. Ker je $Q$ praideal, ki vsebuje $(a)$, mora obstajati nek minimalni praideal $P_i$ nad $(a)$, da velja $(a) \subseteq P_i \subseteq Q$. Po predpostavki je $P_i = (p)$, kjer pa je $p$ praelement v $R$, ker je $R$ domena. Očitno torej $Q$ vsebuje neničelen praelement.
	
	\item Dokažimo še, da če je $R$ UFD, potem je vsak minimalni praideal glavni. To seveda sledi direktno iz prejšnje točke, saj je ideal $(0)$ glavni na trivialen način (ta točka je na nek način poseben primer prejšnje). 
\end{enumerate}

\underline{\textbf{Nal. 2:}}
Naj bo $P$ praideal komutativnega kolobarja $R$ in naj bo $\varphi\colon R \to R_P$ standarden lokalizacijski homomorfizem, definiran s predpisom $r \mapsto \frac{r}{1}$. Označimo $S_P(0) = \ker \varphi$.
\begin{enumerate}[label=(\alph*)]
	\item Dokažimo, da je $S_P(0)$ vsebovan v vsakem $P$-primarnem idealu.
	
	Najprej pokažimo vsebovanost $S_P(0) \subseteq P$. Naj bo $a \in S_P(0)$, torej $\frac{a}{1} = 0$. Potem obstaja element $u \in R\setminus P$, da je $ua = 0.$ Ker očitno $u \notin P$, velja $a \in P$, saj je $P$ praideal.
	
	Sedaj vzemimo poljuben $P$-primaren ideal $Q$, torej $P = \sqrt{Q}$. Posebej velja $Q \subseteq P$. Po lemi 7.21 s predavanj velja $(Q^e)^c = Q$ in $(S_P(0)^e)^c = S_P(0)$. Potem pa velja
	\[
	S_P(0) = (0)^c \subseteq (Q^e)^c = Q.
	\]
	\item Dokažimo še, da je ideal $S_P(0)$ $P$-primaren natanko tedaj, ko je $P$ minimalni praideal.
	
	Recimo, da je $S_P(0)$ $P$-primaren, torej naj bo $\sqrt{S_P(0)} = P$ in vzemimo poljuben $x \in P^e \subideal R_P$. Potem je $x = \frac{p}{s}$ za neka elementa $p \in P$ in $s \in R\setminus P$. Po predpostavki obstaja $n \in \N$, da je $p^n \in S_P(0) = \ker\varphi$, torej $x^n = \frac{p^n}{s^n} = 0$. Sledi, da je $P^e$ vsebovan v nilradikalu kolobarja $R_P$, ki pa je presek vseh praidealov v kolobarju $R_P$, torej je $R_P$ minimalni praideal. Hkrati pa so praideali v $R_P$ v bijekciji s praideali v $R$, ki so vsebovani v $P$, torej je $P^e$ tudi maksimalen ideal v $R_P$. Od tod sledi, da je $P^e$ edini praideal v $R_P$, torej je $P$ minimalni ideal v $R$.
	
	Obratno privzemimo, da je $P$ minimalni praideal v $R$. Potem ima $R_P$ natanko en praideal $P^e$. Od tod sledi, da je nilradikal kolobarja $R_P$ kar enak idealu $P^e$. Torej za vsak $p\in P$ obstaja tak $n\in\N$, da je $\left(\frac{p}{1}\right)^n = 0$. Od tod sledi, da je $p^n \in S_P(0)$, torej $P \subseteq \sqrt{S_P(0)}$. Obratna inkluzija sledi direktno iz $S_P(0) \subseteq P$ (dokaz v prejšnji točki).
	
	Zakaj je $S_P(0)$ primaren? Naj bo $ab \in S_P(0) \subseteq P$. Ker je $P$ praideal, je vsaj eden od njiju gotovo v $P$, naj bo to na primer $b$. Če tudi $a \in P$, je dokaz končan, zato privzemimo $a \notin P$. Ker $ab \in S_P(0)$, obstaja $u \in R\setminus P$, da $uab = 0$, vendar je $a \in R\setminus P$, zato $vb = 0$ za $v = ua \in R\setminus P$. Torej $b \in S_P(0)$, kar pokrije še drugo možnost. S tem je dokaz končan. 
	
	\textbf{Opomba:} Zakaj je nilradikal komutativnega kolobarja enak preseku njegovih praidealov?
	 Naj bo $r$ element nilradikala kolobarja $R$ in naj bo $P \in \Spec R$. Potem $r^n = 0$ za neko število $n \in \N$. Potem je $r \cdot r^{n-1} = 0 \in P$ in sledi $r \in P$ ali $r^{n-1} \in P$, saj je $P$ praideal. Po indukciji na $m \leq n-1$ s ponavljanjem tega postopka za drugi primer dobimo $r^m \in P$ za vsak $m = 1,\dots, n-1$, v posebnem primeru $r \in P$. Ker je bil praideal izbran poljubno, je $r \in P$ za vsak $P \in \Spec R$, torej je nilradikal kolobarja $R$ vsebovan v preseku vseh praidealov.
	 Obratno privzemimo, da $x$ ni element nilradikala kolobarja $R$. Oglejmo si množico
	 \[
	 A = \lbrace J \subideal R ;\; \forall m \in \N\colon x^m \notin J\rbrace
	 \]
	 Ta množica ni prazna, saj $(0) \in A$. Delno jo uredimo z običajno inkluzijo. Vsaka veriga $J_1\subseteq J_2 \subseteq \cdots$ ima naravno zgornjo mejo $J = \bigcup_{j\geq 0} J_j \in A$. Po Zornovi lemi obstaja maksimalen element $M \in A$. Dokažimo, da je $M$ praideal. Naj velja $ab\in M$, vendar $a, b \notin M$. Potem je $M$ strogo vsebovan v $M + (a)$ in $M + (b)$, nobeden od teh dveh idealov pa ni v množici $A$, saj je $M$ njen maksimalni element. Torej obstajata $r, s \in \N$, da je $x^r \in M + (a)$ in $x^s \in M + (b)$. Vendar pa je potem $x^{r+s} = x^rx^s \in M + (ab) = M \in A$, kar je protislovje. Od tod sledi, da če $x$ ni v nilradikalu kolobarja $R$, potem $x$ ni vsebovan v nobenem praidealu. Ekvivalentno je presek vseh praidealov vsebovan v nilradikalu kolobarja $R$.
\end{enumerate}


\end{document}

%% TEMPLATES
% lists
%\begin{enumerate}[label=(\alph*)]
% diagram
%\adjustbox{scale=1, center}{
%	\begin{tikzcd}
%		\R_n \arrow[d, "\varphi_n"] \arrow[r, "\Phi"] & \R_m \arrow[d, "\varphi_m"] \\
%		\R \arrow[r, "\widetilde{\Phi}"] & \R
%	\end{tikzcd}
%}