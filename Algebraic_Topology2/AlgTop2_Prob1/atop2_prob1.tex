\documentclass[a4paper, 12pt]{article}

\usepackage[slovene]{babel}
\usepackage[utf8]{inputenc}
\usepackage[T1]{fontenc}
\usepackage{lmodern}
\usepackage{units}
\usepackage{eurosym}
\usepackage{amsmath}
\usepackage{amssymb}
\usepackage{amsthm}
\usepackage{amsfonts}
\usepackage{mathtools}
\usepackage{graphicx}
\usepackage{color}
%\usepackage{url}
\usepackage{hyperref}
\usepackage{enumerate}
\usepackage{enumitem}
\usepackage{pifont}

% set margin and layout here
\usepackage[margin=0.5in]{geometry}

% commonly used math operators
\DeclareMathOperator{\diam}{diam}
\DeclareMathOperator{\rank}{rank}
\DeclareMathOperator{\im}{im}
\DeclareMathOperator{\Lin}{Lin}
\DeclareMathOperator{\Ann}{Ann}

% commonly used math objects
\newcommand{\D}{\mathbb{D}}
\renewcommand{\S}{\mathbb{S}}
\newcommand{\B}{\mathbb{B}}
\newcommand{\N}{\mathbb{N}}
\newcommand{\Z}{\mathbb{Z}}
\newcommand{\Q}{\mathbb{Q}}
\newcommand{\R}{\mathbb{R}}
\newcommand{\C}{\mathbb{C}}
\renewcommand{\P}{\mathbb{P}}

% commonly used math symbols
\newcommand{\closure}[1]{\overline{#1}}
\newcommand{\subideal}{\vartriangleleft}
\newcommand{\supideal}{\vartriangleright}

% title data - MODIFY
\title{Algebraic Topology 2 - Problem 1}
\author{Benjamin Benčina, 27192018}

\begin{document}

\maketitle
\underline{\textbf{Where does the proof go wrong:}}
We will concentrate on dimensions $2$ and up, since the case $n=1$ is trivial and done right.
Let us first inspect $\beta$, which is supposed to represent the cohomology class of $\alpha^{n-1}$. Note that we chose it to be $1$ on the exterior and $0$ on the interior ($n-1$)-simplices. However, when we look at it a bit closer in $n=2$, something strange happens. As we defined it, it is easy to verify, that $\delta\beta$ of any of the two possible $2$-simplices is equal to $0$, so it is indeed a cocycle. But is it possible that $\beta$ is a coboundary as well, hence being equal to the $0$ cohomology class?
We see that $\beta + \im\delta  = 0 + \im\delta \iff \beta \in \im\delta \iff \exists \gamma \in C^0$ such that $\delta\gamma = \gamma\partial = \beta$. If we define $\gamma(v_1) = 1$ and $\gamma(v_2) = 0$ (or vice versa, also taking identifications into account), we get precisely the desired $\gamma$, hence $\beta \sim 0$.

In dimension $n=3$, the situation is similar. Searching for the appropriate $\gamma$, we see that for $\Z_2$ coefficients, the boundary of a $2$-simplex is precisely the sum of all $1$-simplices bounding it. So, if we define $\gamma \in C^1$ as e.g. $1$ on all equatorial $1$-simplices and $0$ elsewhere, we get the desired property $\beta = \gamma\partial$ (recall that we are constructing each higher dimension with suspension, so equatorial w.r.t. suspension). Hence, $\beta \sim 0$ again.

We begin to see a pattern, as we did the same thing in the $n=2$ case. By induction, $\beta \sim 0$ in all cohomology groups for cases $n \geq 2$ (we take $\gamma$ to be $1$ on all equatorial ($n-1$)-simplices).

What to do now? We saw that $\beta \sim 0$ in all non-trivial cases, yet the products we calculated did not come out to all be equal to $0$. We must be doing something else wrong. Recall that $\beta$ was supposed to represent $\alpha^{n-1}$, so what happens, when we try to calculate the product in $n=2$, where $\alpha \cup \beta \sim \alpha \cup \alpha$. We calculate
\begin{align*}
(\alpha \cup \alpha)\sigma_1 &= \alpha[v_{-2}v_{-1}]\alpha[v_{-1}v_1] = 1\cdot 1 = 1 \\
(\alpha \cup \alpha)\sigma_2 &= \alpha[v_{-1}v_1]\alpha[v_1v_2] = 1\cdot 1 = 1
\end{align*}
Hence, $(\alpha \cup \alpha)(\tau) = 0$, even though $\alpha^2$ is supposed to be a generator of $H^2$. We suspect that there may be something deeper at fault, concretely, the way we even calculate our product.

The cup product is calculated with the use of front and back faces, but they are in turn determined with the choice of our linear ordering or vertices. The cup product does not care about our identifications, it sees only our ordering, therefore we suspect that we did not coordinate our ordering well w.r.t. the identifications on our $\Delta$-complex. Indeed, drawing the case $n=2$ shows that our ordering $v_{-2}<v_{-1}<v_1<v_2$ gives us the fundamental polygon of the torus instead of the real projective plane (we really first notice this when we see that $[v_1v_2]$ is identified with $[v_{-1}v_{-2}]$, which is in the wrong order). The same problem persists in higher dimensions for the exact same reason.
\newline

\underline{\textbf{How to (hopefully) fix the proof:}}
We hope that we may fix the ordering of vertices and obtain the correct structure that agrees well with the identifications made. With a bit of rectangle drawing for $n=2$ we see that the ordering $v_{-1}<v_{1}<v_{-2}<v_2$ gives us precisely the correct fundamental polygon ($v_1<v_{-1}<v_2<v_{-2}$ also works). We calculate
\begin{align*}
(\alpha \cup \alpha)\sigma_1 &= (\alpha \cup \alpha)[v_{-1}v_1v_{-2}] = \alpha[v_{-1}v_1]\alpha[v_1v_{-2}] = 1\cdot 0 = 0 \\
(\alpha \cup \alpha)\sigma_2 &=(\alpha \cup \alpha)[v_{-1}v_1v_{2}] = \alpha[v_{-1}v_1]\alpha[v_1v_{2}] = 1 \cdot 1 = 1
\end{align*}
and hence $(\alpha \cup \alpha)(\tau) = 0 + 1 = 1$. This ordering gives us a satisfactory result.
If we calculate by hand the product $\alpha \cup \alpha^2$ in dimension $n = 3$ with the ordering $v_{-1}<v_{1}<v_{-2}<v_2<v_{-3}<v_3$, we get the same result. Moreover, the product is again non-zero only for the last $3$-simplex $[v_{-1}v_1v_2v_3]$ (we see this forms a good basis for induction later on). Indeed,
\begin{align*}
&(\alpha \cup \alpha^2)[v_{-1}v_1v_{-2}v_{-3}] = \alpha[v_{-1}v_1]\alpha[v_1v_{-2}]\alpha[v_{-2}v_{-3}] = 1 \cdot 0 \cdot * = 0  \\
&(\alpha \cup \alpha^2)[v_{-1}v_1v_{2}v_{-3}] = \alpha[v_{-1}v_1]\alpha[v_1v_{2}]\alpha[v_{2}v_{-3}] = 1 \cdot 1 \cdot 0 = 0 \\
&(\alpha \cup \alpha^2)[v_{-1}v_1v_{-2}v_{3}] = \alpha[v_{-1}v_1]\alpha[v_1v_{-2}]\alpha[v_{-2}v_{3}] = 1 \cdot 0 \cdot * = 0 \\
&(\alpha \cup \alpha^2)[v_{-1}v_1v_{2}v_{3}] = \alpha[v_{-1}v_1]\alpha[v_1v_{2}]\alpha[v_{2}v_{3}] = 1 \cdot 1 \cdot 1 = 1
\end{align*}
and hence $(\alpha\cup\alpha^2)(\tau) = 0 + 0 + 0 + 1 = 1$.

The question now is can we generalize this ordering for arbitrary $n >= 2$. Concretely, let us try to show that the ordering $v_{-1}<v_{1}< \dots < v_{-n}<v_n$ is satisfactory. The proof will of course go by induction.
Suppose the statement holds for all cases in dimensions $\leq n-1$, . Firstly, every $n$-simplex in dimension $n$ has precisely one of $v_{-n}, v_n$ for its vetrtex and by our ordering, it is always in the last place. Secondly, the cup product is graded commutative, but with $\Z_2$ coefficients this is just equal to commutativity, that is $\alpha \cup \alpha^{n-1} = \alpha^{n-1} \cup \alpha$. Now we use, that $\alpha^{n-1}$ is non-zero only for the ($n-1$)-simplex $[v_{-1}v_1\dots v_{n-1}]$ (all positive indices bar the first one) (this is in dimension $n-1$, but recall that here the product is the same by the inductive construction property of (real) projective spaces). By commutativity that means we need only calculate our product on the simplices $[v_{-1}v_1\dots v_{n-1}v_n]$ and $[v_{-1}v_1\dots v_{n-1}v_{-n}]$. Indeed,
\begin{align*}
(\alpha \cup \alpha^{n-1})[v_{-1}v_1\dots v_{n-1}v_n] &= (\alpha^{n-1}\cup\alpha)[v_{-1}v_1\dots v_{n-1}v_n] = \alpha^{n-1}[v_{-1}v_1\dots v_{n-1}]\alpha[v_{n-1}v_n] = \alpha[v_{n-1}v_n] \\
(\alpha \cup \alpha^{n-1})[v_{-1}v_1\dots v_{n-1}v_{-n}] &= (\alpha^{n-1}\cup\alpha)[v_{-1}v_1\dots v_{n-1}v_{-n}] = \alpha^{n-1}[v_{-1}v_1\dots v_{n-1}]\alpha[v_{n-1}v_{-n}] = \alpha[v_{n-1}v_{-n}]
\end{align*}
But by our choice of $\alpha$, precisely one of simplices $[v_{n-1}v_n]$ and $[v_{n-1}v_{-n}]$ will be carried to $1$, concretely (the way we chose it) this will hold for the simplex $[v_{-1}v_1\dots v_{n}]$ (which again justifies our inductive hypothesis). This means that
\[
(\alpha \cup \alpha^{n-1})(\tau) = (\alpha \cup \alpha^{n-1})(\Sigma_i\sigma_i) = 0 + \cdots 0 + 1 = 1,
\]
which concludes the proof for $\Z_2$ coefficients. The rest of the proof does not have mistakes and now holds true.
\end{document}

%% TEMPLATES
%\begin{enumerate}[label=(\alph*)]